\chapter*{Agradecimientos}

Son muchas las personas que te encuentras a lo largo de una tesis. Algunos te apoyan y otros simplemente te hacen el viaje más agradable con su compañía, lo que también vale muchísimo. A todos ellos quiero darles mi agradecimiento. Algunos nombres se me quedarán en el tintero, pero no por ello me olvido de sus portadores. No suelo expresar demasiado y estas cosas me cuestan, pero yo creo que los que me conocen bien saben cuánto los aprecio. 

Los primeros son, por supuesto, mis padres. Sin ellos no estaría donde estoy, no sería lo que soy, no sería a secas. Vosotros me instilasteis la curiosidad que me trajo aquí. Gracias. Os quiero.

A Miguel quiero reservarle también un sitio especial. He aprendido mucho en estos años. Creo que he crecido mucho, y buena parte de ello ha sido por tu ejemplo. 

Ha habido muchos compañeros con los que he compartido muchísimas horas de risas y alguna de desesperación, pero pocas han sido insípidas. Todo el labo MT ha estado ahí cuando hacía falta, pero quería resaltar a Alberto, esa máquina de hacer ciencia que era un poco como un hermano mayor, a Clara con la que me pegaba por la poyata, a Laura que es como una segunda madre, un referente sólido en el que siempre te puedes apoyar cuando no tienes ni idea de por qué tu experimento no sale y que reparte ternura. A Joanna, ese cripto-pedazo de pan. A Susana, por la que tengo un cariño especial, a Bea, Irene, Félix, a todos los demás: un placer.

Gracias a todos los de alrededor, que dan una vidilla que en otros centros no puedo imaginar que tengan: la Cañón encantadora, Teresa con su genio, Cris, Julio, Héctor, Guille, Gaetano al que ahora mismo oigo de fondo (cómo no), Juli, Marcos, Vero, Claudio\ldots a Virginia también: sin tu atención al detalle vete a saber cómo la habría liado con las colonias. Roisin, cómo hemos sudado con el OPT, ¿eh?

Gracias a mis compañeros de máster, esos hermanos de sangre que hemos ido creciendo juntos: Edu y Alejo con los brunchcrafts y todo lo demás, Cris otra vez, Pilar, Inés, Jaime, Miguel, Lucía\ldots Me acuerdo mucho de ese año tan especial, y del camino que hemos hecho desde entonces.

Gracias a mis queridos amigos de la carrera: hubo un momento en que me quedé algo descolgado, pero no me olvido de vosotros, y cuando nos volvemos a ver siempre es un poco igual que siempre. Yuri, compi en el labo y compi en casa, Ana que se casó ya (que fuerte), Tefi de alegría contagiosa, Albertito cuyo blog leo aunque no comente mucho\ldots os quiero. 

Espero veros a todos muchas veces más en lo que me queda, y que podamos seguir compartiendo momentos.

\thispagestyle{cleared}