\chapter{Discussion}
\label{chp:discussion}

The results presented here represent the first \ac{GW} view of \ac{TALE} family proteins \acp{BS} in the vertebrate embryo. As such, they deepen our understanding of the target sequences and modes of action of these Hox cofactors, and raise interesting questions that will have to be addressed in further studies. 

\section{Binding Preferences}

Despite their very similar binding behaviour \textit{in vitro}, Prep and Meis have shown radically different binding sequence preferences \textit{in vivo} in our study. 

\subsection{Location}

This study is the first detailed view and comparison of \ac{TALE} proteins. Only a few \textit{in vivo} target sites were known previously. These have been invariably been regulatory sites located near genes. In contrast, our unbiased \ac{GW} search has revealed striking differences in \ac{TALE} protein promoter and enhancer occupancy. 

Meis binds thousands of extremely conserved gene-remote regions, a high proportion of which show enhancer chromatin marks in comparable cells. It seems reasonable to conclude that many of these sites are enhancers, at least some of which are inactive but 'poised'. Poised enhancers have been suggested to represent the range of responses available to a cell (\cite{Creyghton2010}) and could represent a developmental path restriction by the \ac{TALE} system which would be later narrowed down by the participation in \ac{TALE}-Hox complexes of subsets of Hox proteins available in different cells. 

Along this line, the Pbx-Meis heterodimer has been proposed to function as a 'pioneer' factor that can bind heterochromatin and recruit MyoD to some of its targets, such as myogenin or M-cad (\cite{Berkes2004}). It is notable though that our \ac{CHIP}-seq shows binding of Pbx and Prep at the M-cadherin promoter, but not binding of Meis. [discutir]

% M-cad expressed E11.5 in proximal limb

\subsection{Sequences}


\section{TALE Proteins as Hox cofactors}

\subsection{OCTA Variant Preferences}

\subsubsection{New Hox Target Sequence}

\subsection{Restricted Binding in the Hox Clusters}

%\section{TALE Proteins as Transcription Factors}


\section{Additional Binding Partners}