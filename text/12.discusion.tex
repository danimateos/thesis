\chapter{Discussion}
\label{chp:discussion}

The results presented here represent the first near-comprehensive \ac{GW} view of \ac{TALE} family proteins \acp{BS} in the vertebrate embryo. As such, they deepen our understanding of the target sequences and modes of action of these Hox cofactors, and raise interesting questions that will have to be addressed in further studies. 

\section{Meis vs Prep}

To date, the general picture revealed by \textit{in vitro} studies of Meis and Prep molecular function has revealed very few differences between them (see for example \textcite{Moens2006}, where they are considered interchangeably). Both can bind Pbx directly (\cite{Knoepfler1997, Shen1997}), both recognize very similar sequences (\cite{Knoepfler1997, Chang1997}), both can participate in trimeric complexes with Pbx and Hox proteins (\cite{Berthelsen1998, Shen1999}), and both can activate the transcription of promoters containing their target sequences (\cite{Berthelsen1998, Jacobs1999}). Only two clear differences have been established: the first is that Meis is resposive to \ac{TSA} and \ac{PKA} signalling (\cite{Huang2005}) while Prep is not. The second is that Meis is able to bind some Hox proteins directly without the participation of Pbx, while Prep is not (\cite{Williams2005}).

However, their general biological functions are quite different. The Meis mutant has hematopoietic and developmental phenotypes. Prep, in contrast, is critical at a much earlier stage than Meis in functions apparently more basal, although the Prep1 hypomorphic mutant survives to E17.5, when it dies with a phenotype similar but distinct to that of the Meis1 mutant (\cite{Ferretti2006, DiRosa2007}). One partial explanation of this functional difference is that while the Prep genes are expressed ubiquitously, Meis genes have restricted and dynamic expression domains. However, if they were completely interchangeable the expression of Meis in territories that already express Prep should be redundant. This is clearly not the case. % [expandir contraste fenotipos]

The work described here provides another foundation for explaining the functional differences between Meis and Prep. The near-comprehensive catalogue of \ac{TALE} factor binding sites will be a useful resource in itself for us and other groups when considering mechanisms and mediators of Meis, Prep and Pbx effects. 

The \textit{in vivo} binding sequence preferences we have  described will guide the understanding of \ac{TALE} protein control of expression of downstream genes. A \ac{DECA} or \ac{DECAEXT} sequence found to act in a promoter should be suspected primarily of being bound by a Pbx-Prep complex. Functional \ac{OCTA} sequences located in enhancers should be considered as likely candidates for regulation by Meis and Hoxes, with or without the participation of Pbx. 

\section{Binding Preferences}

Despite their very similar binding behaviour \textit{in vitro}, Prep and Meis have shown radically different binding sequence preferences \textit{in vivo} in our study. 

\subsection{Location}

This study is the first detailed view and comparison of \ac{TALE} proteins. Only a few \textit{in vivo} target sites were known previously. These have been invariably been regulatory sites located near genes. In contrast, our unbiased \ac{GW} search has revealed striking differences in \ac{TALE} protein promoter and enhancer occupancy. 

Meis binds thousands of extremely conserved gene-remote regions, a high proportion of which show enhancer chromatin marks in comparable cells. It seems reasonable to conclude that many of these sites are enhancers, at least some of which are inactive but 'poised'. Poised enhancers have been suggested to represent the range of responses available to a cell (\cite{Creyghton2010}) and could represent a developmental path restriction by the \ac{TALE} system which would be later narrowed down by the participation in \ac{TALE}-Hox complexes of subsets of Hox proteins available in different cells. 

Along this line, the Pbx-Meis heterodimer has been proposed to function as a 'pioneer' factor that can bind heterochromatin and recruit MyoD and the estrogen receptor to some of their targets, such as myogenin or M-cad (\cite{Berkes2004,Magnani2011}). It is notable though that our \ac{CHIP}-seq shows binding of Pbx and Prep at the M-cadherin promoter, but not binding of Meis. % [discutir]

% M-cad expressed E11.5 in proximal limb

In remarkable contrast with \ac{mepb} peaks, \ac{pbpr} peaks are very strongly associated with promoters. This preference is driven by Prep, since \ac{prepe} but not \ac{pbxe} peaks show strong association with promoters.  

\subsection{Sequences}



\subsection{TALE Proteins as Hox cofactors}

The surprising finding that the most frequent sequence motif in Meis peaks is \ac{OCTA}, the Pbx-Hox target site, rather than \ac{HEXA}, the Meis monomer binding site, suggests that most Meis binding sites are trimeric Pbx-Hox-Meis complex binding sites in which Meis does not contact DNA directly. 

% posibilidad de que el anti-Pbx1 no detecte complejos trimericos? esta dirigido a aas de n-ter, podria ser. Se ha usado en EMSA contra prots purificadas?

The large overlap between Meis peaks and published Hox binding sites reinforces the view that Meis is an important Hox cofactor. 

If most Meis peaks with the \ac{OCTA} motif are indeed Hox \acp{BS}, this work represents the first \ac{GW} catalog of \textit{in vivo} \acp{BS} from several different Hox paralog groups. Since to date only two ChIP-seq datasets for vertebrate Hox proteins have been published (\cite{Jung2010, Donaldson2012}) and very few of their regulated genes are known, it could represent an extremely valuable asset for understanding the missing link between Hox protein expression and control of morphogenesis (\cite{ref}).

\subsubsection{OCTA Variant Preferences}

The detailed analysis of \ac{OCTA} motif variants reveals preferences in the variable dinucleotide that match what we know about Pbx-Hox preferences and the ability of Pbx and Meis to bind subsets of the Hox proteins. 

In \ac{meise} peaks, the most overrepresented variant is TGATTTAT. This particular \ac{OCTA} variant is the preferred binding target for AbdB-related Hoxes, which are precisely those that can bind Meis without the participation of Pbx. In contrast, the most common variant in \ac{mepb} peaks is TGATTGAT, which is the preferred variant of 3' Hoxes, which can't bind Meis directly, in which case it has to rely on indirect binding to Hoxes via Pbx.

This argues for differential use of cofactors among different paralog groups of Hox proteins \textit{in vivo}. 

\subsubsection{New Hox Target Sequence}

We have found an \ac{OCTA} variant, TGATGAAT, that had not been described to be bound by Hoxes. Its overrepresentation in several \ac{TALE} peak subsets suggested it could represent a novel Hox binding target. We have confirmed this by \ac{EMSA} experiments. [completar]



\subsection{Restricted Binding in the Hox Clusters}



\section{Additional Binding Partners}

We have discovered four new sequences not shown previously to be involved in \ac{TALE} protein binding. 

The sequence CWGSCWG is the only Pbx core motif we found. It appears in the \ac{meise} and \ac{pbxe} subsets and a similar one appears in \ac{mepb} peaks. It is likely to represent the binding sequence for an unknown Pbx or Pbx-Meis cofactor. It is interesting that highly similar sequences have been reported previously as unidentified motifs in several hematopoietic factor ChIP-seq experiments: An Scl/Tal1 ChIP-seq in immature erythocytes (\cite{Kassouf2010}), a 10-factor ChIP-seq in the hematopoietic progenitor cell line HPC-7 (\cite{Wilson2010}), and a \ac{GW} analysis of Gata1, Gata2, Runx1, Fli1, and Scl/Tal1 binding in primary human megakaryocytes (\cite{Tijssen2011}). The only common factor to these three studies is Scl/Tal1. Scl/Tal1 is a \ac{BHLH} protein and thus binds an E-box. Although this novel motif resembles an E-box, the binding preferences of Scl/Tal1 are well characterized and do not include this variant, in which there is an additional nucleotide between the two halves of a canonical E-box. % TGIF? paper de binding alterno y efecto opuesto.

Scl/Tal1 can interact with Gata1 to specify erythroid cells (\cite{ref}) and we have found an Scl/Tal1-related motif but not any motif resembling Gata1. Considering that Meis is necessary for megakaryocyte generation in the mouse embryo, it is tempting to speculate that Scl might be recruited by Gata1 in erythroid cells (\cite{Anguita2004}) and Meis1 in megakaryocytes. These two cell types share a precursor cell, the \ac{MEP} (\cite{ref}), so this could represent a protein recruitment-based developmental switch. Scl/Tal1 mutants die at 9.5 due to complete lack of blood (\cite{ref}), but deletion in adult mice shows it is necessary for maintenance only of the erythroid and megakaryocytic lineages (\cite{Mikkola2003}), which points to dual roles in embryonic \acp{HSC} and \acp{MEP}. It is remarkable that a DNA-binding deficient form of Scl rescues the early hematopoiesis-related lethality and allows the embryo to progress to E14.5, when it dies with incomplete maturation of erythrocytes (\cite{Porcher1999, Kassouf2008}). This probably means some of its functions are mediated by cofactors that allow it to bind DNA independently of its own DNA-binding domain. Scl is also expressed in the neural tube in a \ac{DV}-restricted manner (Muroyama2005), as is Meis.



% Wilson2009: Scl/Tal1 binding shows Meis motifs in Hematopoietic precursor cell-7 (HPC-7) line
% Kassouf2010: immature, Ter119− erythroid cell populations derived from day E12.5 wild-type (Tal1WT/WT) fetal livers 

The CCCgCCC sequence is an accessory motif found in all peaks subsets in which Prep is present. It exactly matches the GC-box, a promoter component that is the binding site for Sp1 (\cite{Kriwacki1992}) and its related factors Sp3, Sp4 and the KLF family. The defining characteristic of the Sp/Klf family is a \ac{C-ter} DNA-binding domain containing three C2H2 Zinc finger motifs. A short motif \ac{N-ter} to the Zinc fingers, called the Buttonhead (btd) box, is present in Sp proteins but not in Klfs (\cite{Suske2005}). It does not contain a tryptophan. It is remarkable that Sp genes have a conserved genomic organization: they are linked two by two and 3 of these 4 syntenyc pairs are linked to Hox genes. Additionaly, 5 of the 9 Sp genes in the mouse genome have \ac{TALE} factor peaks in their promoters, most often \ac{pbpr} peaks. The coincidence of Pbx1-Prep with an Sp target sequence and possible control of Sp genes by Prep and Pbx suggests these two families act as a previously unrecognized functional module. Deeper study would be needed to ascertain the nature of the relationship, if it exists. However, a computational analysis of human promoters found the motif for Sp1 \parencite{Hartmann2013}, so its presence in Prep subsets could be a byproduct of their high proportion in promoters.

% [Sps in RNAseq? they change but not much. Max logFC = 1,52 (Sp7)] 

The CCAAT sequence appears in \ac{pbpr} peaks. It is a known component of promoters and enhancers, bound by the NF-Y and the CEBP proteins. [completar] As with the GC-box, it is not surprising to find a promoter-related sequence in a subset of peaks that are overwhelmingly associated with promoters, and indeed this sequence also appears in a general motif search in promoters \parencite{Hartmann2013}. However, in this case it is clear that there is a specific interaction between the Pbx1-Prep heterodimer and a CCAAT-binding factor, since \ac{DECA} and the CCAAT box appear at a fixed distance within the \ac{DECAEXT} motif. The CCAAT box is present in many promoters of genes overexpressed in cancer (\cite{Dolfini2013}). The NFY heterotrimer might itself be transcriptionally regulated by the \ac{TALE} Hox cofactors, since the NFY-C gene (coding for the $\gamma$ subunit) has Meis and Prep peaks with several \ac{DECAEXT} motifs within its first intron. In fact, the Prep peaks actually overlap the alignment point for the \ac{TSS} of a rat isoform of NFY-C that has not yet been described to exist in the mouse, and show LICR \ac{MEF} promoter (RNApolII\textsuperscript{+} H3K4Me3\textsuperscript{+}) marks. Very obviously, this fact suggests the existence of a previously unrecognized isoform of NFY-C in the mouse and its specific regulation by the \ac{TALE} proteins through differential promoter recruitment. NFY-C is downregulated under the significance threshold in the \ac{prepi} embryo, but this does not rule out isoform balance effects or redundant control by Prep1/2. This is an interesting avenue to pursue regarding the search for mediators of the involvement in cell proliferation and cancer of Meis and Prep.

% CEBP$\alpha$ downregulation is involved in AML and its upregulation is implicated in ALL (\cite{Paz-Priel2011})

 %Klf? Sp1? GC box. Sp1 binds the estrogen receptor and CEBPB. Zhao and Meng 2005: most interesting phenotypes Sp3 (hematop dev) and Sp6, Sp8 (limbs). Sp2 recognizes GT box rather than GC box. Sp1 and Sp3 same target. knockout of KLF1 results in selective defects in erythropoiesis [47,48], whereas KLF2 is involved specifically in T-cell quiescence and survival (Kaczynski et al 2003). Many Sps have prep/pbx in their promoters! wnt --> Sp6/8 --> Fgf8 in AER. btd (Sp1 Drosophila homolog) is a head segmentation gene. 
 
% overexpression of Pu.1 results in maturation block with an overrepresentation of immature erythroblasts. Zhang et al. showed by EMSA (electrophoretic mobility shift assays) that PU.1 directly interacts with the Gata-1 DNA recognition motif thereby blocking its DNA-binding activity

% Bjerke et al 2011: Cooperative Transcriptional Activation by Klf4, Meis2, and Pbx1

Poly-purine tract - Pu.1? triplexes?
% FANCM, helicase, found in meis chip-prot, promotes dissociation of DNA triplexes

\section{ejemplo(s?) de cherry picking}

Id2 (inhibitor of differentiation 2) has many \ac{TALE} binding sites in its vicinity, including a \ac{pbpr} peak in its promoter. Id2 represses myogenesis and it is in turn repressed by Rp58 (also known as Zfp238), a direct target of MyoD, to promote myogenesis (\cite{Yokoyama2009}). Rp58 has a \ac{pbpr} peak 2kb upstream of its TSS plus an hematopoietic Meis peak from \cite{Wilson2010}. It is suggestive that this system has already shown to be connected with the \ac{TALE} proteins (see introduction), which suggests a relatively widespread intertwining of \ac{TALE} function and MyoD. This example exemplifies the potential usefulness of the data we have generated to give insight into unexpected functions and relationships of the \ac{TALE} genes and to generate hypotheses. 

% *Id1 prep peak with HEXA and GC box motifs