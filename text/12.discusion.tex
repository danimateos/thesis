\chapter{Discussion}
\thispagestyle{cleared}
\cleardoublepage
\label{chp:discussion}

The results presented here represent the first near-comprehensive \ac{GW} view of \ac{TALE} family proteins \acp{BS} in the vertebrate embryo. As such, they deepen our understanding of the target sequences and modes of action of these Hox cofactors, and raise interesting questions that will have to be addressed in further studies. 

\section{Meis vs Prep}

To date, the general picture revealed by \textit{in vitro} studies of Meis and Prep molecular function has revealed very few differences between them (see for example \textcite{Moens2006}, where they are considered interchangeably). Both can bind Pbx directly (\cite{Knoepfler1997, Shen1997}), both recognize very similar sequences (\cite{Knoepfler1997, Chang1997}), both can participate in trimeric complexes with Pbx and Hox proteins (\cite{Berthelsen1998, Shen1999}), and both can activate the transcription of promoters containing their target sequences (\cite{Berthelsen1998, Jacobs1999}). Only two clear differences at the molecular level have been established: the first is that Meis is resposive to \ac{TSA} and \ac{PKA} signalling (\cite{Huang2005}) while Prep is not. The second is that Meis is able to bind some Hox proteins directly without the participation of Pbx, while Prep is not (\cite{Williams2005}).

However, their general biological functions are quite different. The Meis mutant has hematopoietic and developmental phenotypes. Prep, in contrast, is critical at a much earlier stage than Meis in functions apparently more basal, although the Prep1 hypomorphic mutant survives to E17.5, when it dies with a phenotype similar but distinct to that of the Meis1 mutant \parencite{Ferretti2006, DiRosa2007}. One partial explanation of this functional difference is that while the Prep genes are expressed ubiquitously, Meis genes have restricted and dynamic expression domains. However, if they were completely interchangeable the expression of Meis in territories that already express Prep should be redundant. This is clearly not the case. Indeed, regarding their involvement in cancer, Meis and Prep actually have effects in opposite directions: Meis is a proto-oncogene that can induce leukemias by over-expression, while Prep is a tumor suppressor gene \parencite{Moskow1995, Thorsteinsdottir2001, Iotti2011, Longobardi2010, Wong2007}. % [expandir contraste fenotipos]

The work described here provides another foundation for explaining the functional differences between Meis and Prep. The near-comprehensive catalogue of \ac{TALE} factor binding sites will be a useful resource in itself for us and other groups when considering mechanisms and mediators of Meis, Prep and Pbx effects. 

The \textit{in vivo} binding sequence preferences we have  described will guide the understanding of \ac{TALE} protein control of expression of downstream genes. A \ac{DECA} or \ac{DECAEXT} sequence found to act in a promoter should be suspected primarily of being bound by a Pbx-Prep complex. Functional \ac{OCTA} sequences located in enhancers should be considered as likely candidates for regulation by Meis and Hoxes, with or without the participation of Pbx. 


Although only a few of the peaks we describe are triple Meis-Prep-Pbx binding sites, they could represent the points of input into jointly regulated systems. Meis and Prep could be collaborating or antagonizing each other. The fact that in almost all of the triple peaks we have assayed there is no simultaneous binding of Meis and Prep means that, in most cases, Meis and Prep compete for binding to the same sequences. In practice that would mean Meis displaces the ubiquitous and pre-existing Prep in some or all of the cells in which it is expressed. 

\section{Binding Preferences}

Despite their very similar binding behaviour \textit{in vitro}, Prep and Meis have shown radically different binding sequence preferences \textit{in vivo} in our study. 

This study is the first detailed view and comparison of \ac{TALE} protein \textit{in vivo} \acp{BS}. Only a few target sites were known previously. These have been invariably been regulatory sites located near genes. In contrast, our unbiased \ac{GW} search has revealed striking differences in \ac{TALE} protein promoter and enhancer occupancy. 

Meis binds thousands of extremely conserved gene-remote regions, a high proportion of which show enhancer chromatin marks in comparable cells. It seems reasonable to conclude that many of these sites are enhancers, at least some of which are inactive but "poised". Poised enhancers have been suggested to represent the range of responses available to a cell (\cite{Creyghton2010}) and could represent a developmental path restriction by the \ac{TALE} system which would be later narrowed down by the participation in \ac{TALE}-Hox complexes of subsets of Hox proteins available in different cells. 

Along this line, the Pbx-Meis heterodimer has been proposed to function as a "pioneer" factor that can bind heterochromatin and recruit MyoD and the estrogen receptor to some of their targets, such as myogenin or M-cad \parencite{Berkes2004, Magnani2011}. It must be noted though that our \ac{CHIP}-seq shows binding of Pbx and Prep at the M-cadherin promoter, but not binding of Meis. M-cadherin is expressed at E11.5 in the proximal limb. Meis could be indirectly responsible for this restriction, even if it does not bind the M-cadherin promoter at that stage, by making Pbx available at the nucleus for Pbx-Prep complex formation. It could also bind heterochromatin in complex with Pbx and be replaced by Prep afterwards.  %[quitar lo de M-cad?]

In remarkable contrast with \ac{mepb} peaks, \ac{pbpr} peaks are very strongly associated with promoters. This preference is driven by Prep, since \ac{prepe} but not \ac{pbxe} peaks show strong association with promoters.  

The transcriptional effects of Meis and Prep are often in opposing directions on the same gene, despite the fact that they rarely bind the same cis-regulatory sequences.%[argue]

While \textit{in vitro} both Meis and Prep show the ability to bind DNA with Pbx as heterodimers, we have found that, \textit{in vivo}, Prep is much more likely to do so than Meis. Less than 15\% of Meis peaks show a \ac{DECA} motif, the target for this kind of heterodimers. Conversely, while both Meis and Prep are able to form triple complexes with Pbx and Hoxes \textit{in vitro}, only Meis seems to do so \textit{in vivo}: the proportion of \ac{prepe} and \ac{pbpr} peaks that show the \ac{OCTA} motif is under 5\%. 

This, together with the clear divide in binding location between Meis and Prep peaks, paints a picture in which Meis and Prep are functionally specialized, with Meis participating almost exlusively in the Pbx-Hox system by binding extremely conserved enhancers in \ac{TSS}-remote positions and Prep binding promoters in concert with Pbx but independently of Hoxes. It is remarkable that no other study has detected such a clear division of labour between Meis and Prep.

\section{TALE Proteins as Hox cofactors}

The surprising finding that the most frequent sequence motif in Meis peaks is \ac{OCTA}, the Pbx-Hox target site, rather than \ac{HEXA}, the Meis monomer binding site, or \ac{DECA}, the Pbx-Meis/Prep target, suggests that most Meis binding sites are trimeric Pbx-Hox-Meis complex binding sites in which Meis does not contact DNA directly. The large overlap between Meis peaks and published Hox binding sites reinforces the view that Meis is an important Hox cofactor. \textit{In vitro} experiments show Hox proteins can bind \ac{OCTA}-containing Meis peak sequences. 

% posibilidad de que el anti-Pbx1 no detecte complejos trimericos? esta dirigido a aas de n-ter, podria ser. Se ha usado en EMSA contra prots purificadas?

If most Meis peaks with the \ac{OCTA} motif are indeed Hox \acp{BS}, this work represents the first \ac{GW} catalog of \textit{in vivo} \acp{BS} from several different Hox paralog groups. Since to date only two ChIP-seq datasets for vertebrate Hox proteins have been published \parencite{Jung2010, Donaldson2012} and very few of their downstream genes are known, it could represent an extremely valuable asset for understanding the missing link between Hox protein expression and control of morphogenesis. %[change wording? cite ref]

\subsection{OCTA Variant Preferences}

The detailed analysis of \ac{OCTA} motif variants reveals preferences in the variable dinucleotide that match what we know about Pbx-Hox preferences and the ability of Pbx and Meis to bind subsets of the Hox proteins. 

In \ac{meise} peaks, the most overrepresented variant is TGATTTAT. This \ac{OCTA} variant is the preferred binding target for AbdB-related Hoxes, which are precisely those that can bind Meis without the participation of Pbx. In contrast, the most common variant in \ac{mepb} peaks is TGATTGAT, which is the preferred variant of 3' Hoxes. These can not bind Meis directly, so Meis has to rely on indirect binding to Hoxes via Pbx. However, as mentioned before, it is hard to extract conclusions on the relative participation of different Hox paralog groups in Meis-Pbx1-Hox complexes because the abundance of Hoxes in our sample is uneven. %The change in relative sequence preferences observed between \ac{meise} and \ac{mepb} is interesting. 

We have found an \ac{OCTA} variant, TGATGAAT, that had not been described to be bound by Hoxes. Its overrepresentation in several \ac{TALE} peak subsets suggested it could represent a novel Hox binding target. We have confirmed this by \ac{EMSA} experiments. 

\section{Restricted Binding in the Hox Clusters}

It was unexpected that the \ac{TALE} Hox cofactors we have inspected would have a pattern of binding in the Hox clusters that is tightly restricted to their 3' part. %[completar, relacionar con el modelo de rosario]

% Soshnikova and Duboule 2009 dicen que la organizacion, la cromatina y la transcripci'on van de la mano, pero no es asi segun sus figuras.
Additionally, Meis peaks extend past the 3', but not 5', end of all Hox clusters and into the adjacent gene deserts. These have been shown to contain cis-regulatory elements and to establish long-range chromatin interactions with the Hox clusters themselves \parencite{Noordermeer2011, Andrey2013}. In all Hox clusters, the range of these regulatory interactions coincides with the range of Meis peaks extending past the end of the clusters. Consistently, we find no Meis peaks in the gene desert 5' to the HoxD cluster that is known to contain many regulatory elements important for the expression of HoxD genes in the digits \parencite{Montavon2011}, where Meis is not expressed. This "regulatory archipelago" establishes long-range contacts with the 5' region of the HoxD cluster up to HoxD9, the 3' most HoxD gene expressed in the digits. 

Still, there is an unexplained aspect of \ac{TALE} binding in the Hox clusters. The transcriptional effect of Meis and Prep is strongest in the 5' part of the clusters, where they do not bind. Silent Hox clusters form discrete inactive chromatin compartments, and upon activation they adopt a chromatin conformation separated in an active and an inactive compartment \parencite{Noordermeer2011}. The boundary between the topologically associated domains evolves over time, providing a mechanistic basis for temporal colinearity. This proposed mechanism also provides a possible explanation for Meis and Prep effects on the transcription of parts of the Hox cluster to which they do not bind: they could be influencing this collinearity mechanism by serving as anchors between the 3' part of the clusters and its associated gene desert. 

%[meter una explicacion plausible de porque los hoxes suben en meis ko]

% Montavon and Duboule 2013:  Hoxd gene activation in other embryonic structures (proximal digits, digestive tract) also relies on long-range controls [in the telomeric gene desert], yet the corresponding regulatory elements have not yet been identified (dashed arrows). Andrey et al 2013: elements in the HoxD telomeric gene desert. CNS39, with limb activity, contains a MEis peak in eye but not trunk chipseq.
%[Indirect effect on transcription?]
%[HoxA9-13 no responden a RA]
%Ferraiuolo 2010: cambio de conformacion en resupesta a RA. "chromatin remodeling does not appear restricted to the region induced by RA..." ---> related to indirect effect of Meis KO?

\section{Additional Binding Partners}

We have discovered four new sequences not shown previously to be involved in \ac{TALE} protein binding. 

The sequence CWGSCWG is the only Pbx core motif we found. It appears in the \ac{meise} and \ac{pbxe} subsets and a similar one appears in \ac{mepb} peaks. It is likely to represent the binding sequence for an unknown Pbx or Pbx-Meis cofactor. It is interesting that highly similar sequences have been reported previously as unidentified motifs in several hematopoietic factor ChIP-seq experiments: An Scl/Tal1 ChIP-seq in immature erythocytes (\cite{Kassouf2010}), a 10-factor ChIP-seq in the hematopoietic progenitor cell line HPC-7 (\cite{Wilson2010}), and a \ac{GW} analysis of Gata1, Gata2, Runx1, Fli1, and Scl/Tal1 binding in primary human megakaryocytes (\cite{Tijssen2011}). The only common factor to these three studies is Scl/Tal1. Scl/Tal1 is a \ac{BHLH} protein and thus binds an E-box. Although this novel motif resembles an E-box, the binding preferences of Scl/Tal1 are well characterized and do not include this variant, in which there is an additional nucleotide between the two halves of a canonical E-box. % TGIF? paper de binding alterno y efecto opuesto.

Scl/Tal1 can interact with Gata1 to specify erythroid cells (\cite{ref}) and we have found an Scl/Tal1-related motif but not any motif resembling Gata1. Considering that Meis is necessary for megakaryocyte generation in the mouse embryo, it is tempting to speculate that Scl might be recruited by Gata1 in erythroid cells (\cite{Anguita2004}) and Meis1 in megakaryocytes. These two cell types share a precursor cell, the \ac{MEP} (\cite{ref}), so this could represent a protein recruitment-based developmental switch. Scl/Tal1 mutants die at 9.5 due to complete lack of blood (\cite{ref}), but deletion in adult mice shows it is necessary for maintenance only of the erythroid and megakaryocytic lineages (\cite{Mikkola2003}), which points to dual roles in embryonic \acp{HSC} and \acp{MEP}. It is remarkable that a DNA-binding deficient form of Scl rescues the early hematopoiesis-related lethality and allows the embryo to progress to E14.5, when it dies with incomplete maturation of erythrocytes (\cite{Porcher1999, Kassouf2008}). This probably means some of its functions are mediated by cofactors that allow it to bind DNA independently of its own DNA-binding domain. Scl is also expressed in the neural tube in a \ac{DV}-restricted manner (Muroyama2005), as is Meis.



% Wilson2009: Scl/Tal1 binding shows Meis motifs in Hematopoietic precursor cell-7 (HPC-7) line
% Kassouf2010: immature, Ter119− erythroid cell populations derived from day E12.5 wild-type (Tal1WT/WT) fetal livers 

The CCCgCCC sequence is an accessory motif found in all peaks subsets in which Prep is present. It exactly matches the GC-box, a promoter component that is the binding site for Sp1 (\cite{Kriwacki1992}) and its related factors Sp3, Sp4 and the KLF family. The defining characteristic of the Sp/Klf family is a \ac{C-ter} DNA-binding domain containing three C2H2 Zinc finger motifs. A short motif \ac{N-ter} to the Zinc fingers, called the Buttonhead (btd) box, is present in Sp proteins but not in Klfs (\cite{Suske2005}). It does not contain a tryptophan. It is remarkable that Sp genes have a conserved genomic organization: they are linked two by two and 3 of these 4 syntenyc pairs are linked to Hox genes. Additionaly, 5 of the 9 Sp genes in the mouse genome have \ac{TALE} factor peaks in their promoters, most often \ac{pbpr} peaks. The coincidence of Pbx1-Prep with an Sp target sequence and possible control of Sp genes by Prep and Pbx suggests these two families might act as a previously unrecognised functional module. Deeper study would be needed to ascertain the nature of the relationship, if it exists. However, a computational analysis of human promoters found the motif for Sp1 \parencite{Hartmann2013}, so its presence in Prep subsets could be a byproduct of their high proportion in promoters.

% [Sps in RNAseq? they change but not much. Max logFC = 1,52 (Sp7)] 

The CCAAT sequence appears in \ac{pbpr} peaks. It is a known component of promoters and enhancers, bound by the NF-Y and the CEBP proteins. As with the GC-box, it is not surprising to find a promoter-related sequence in a subset of peaks that are overwhelmingly associated with promoters, and indeed this sequence also appears in a general motif search in promoters \parencite{Hartmann2013}. However, in this case it is clear that there is a specific interaction between the Pbx1-Prep heterodimer and a CCAAT-binding factor, since \ac{DECA} and the CCAAT box appear at a fixed distance within the \ac{DECAEXT} motif. The CCAAT box is present in many promoters of genes overexpressed in cancer (\cite{Dolfini2013}). The NFY heterotrimer might itself be transcriptionally regulated by the \ac{TALE} Hox cofactors, since the NFY-C gene (coding for the $\gamma$ subunit) has Meis and Prep peaks with several \ac{DECAEXT} motifs within its first intron. In fact, the Prep peaks actually overlap the alignment point for the \ac{TSS} of a rat isoform of NFY-C that has not yet been described to exist in the mouse, and show LICR \ac{MEF} promoter (RNApolII\textsuperscript{+} H3K4Me3\textsuperscript{+}) marks. Very obviously, this fact suggests the existence of a previously unrecognized isoform of NFY-C in the mouse and its specific regulation by the \ac{TALE} proteins through differential promoter recruitment. NFY-C is downregulated under the significance threshold in the \ac{prepi} embryo, but this does not rule out isoform balance effects or redundant control by Prep1/2. This is an interesting avenue to pursue regarding the search for mediators of the involvement in cell proliferation and cancer of Meis and Prep.

% CEBP$\alpha$ downregulation is involved in AML and its upregulation is implicated in ALL (\cite{Paz-Priel2011})

 %Klf? Sp1? GC box. Sp1 binds the estrogen receptor and CEBPB. Zhao and Meng 2005: most interesting phenotypes Sp3 (hematop dev) and Sp6, Sp8 (limbs). Sp2 recognizes GT box rather than GC box. Sp1 and Sp3 same target. knockout of KLF1 results in selective defects in erythropoiesis [47,48], whereas KLF2 is involved specifically in T-cell quiescence and survival (Kaczynski et al 2003). Many Sps have prep/pbx in their promoters! wnt --> Sp6/8 --> Fgf8 in AER. btd (Sp1 Drosophila homolog) is a head segmentation gene. 
 
% overexpression of Pu.1 results in maturation block with an overrepresentation of immature erythroblasts. Zhang et al. showed by EMSA (electrophoretic mobility shift assays) that PU.1 directly interacts with the Gata-1 DNA recognition motif thereby blocking its DNA-binding activity

% Bjerke e t al 2011: Cooperative Transcriptional Activation by Klf4, Meis2, and Pbx1

The poly-purine motif is interesting because it might be the binding site for an unknown partner of Meis, but it also could represent a sequence pattern that does not bind a particular protein but results in structural features favouring Meis binding. Poly-purine tracts have been shown to be able to form DNA triple helical structures \textit{in vitro} (\cite{ref}), to be implicated in splicing (\cite{ref}), and are a known component of retroviruses (\cite{ref})%[completar, explorar, fill refs].

%Poly-purine tract - Pu.1? triplexes?
% FANCM, helicase, found in meis chip-prot, promotes dissociation of DNA triplexes

\section{Meis Contexts}

The comparison of Meis binding sites in the trunk, eye and \acp{HSPC} provides an interesting frame in which to evaluate the significance of some aspects of Meis trunk binding sites. It is clear that Meis eye and \ac{HSPC} binding sites are more similar in general to each other than to trunk binding sites, despite trunk and \ac{HSPC} peaks overlapping the most of the three possible pairings: both core motif composition and conservation values point to trunk binding sites being apart from the other two sets. One aspect that could be related to this is that compared to those tissues, our sample was much more heterogeneous. This must have biased the detection in favour of binding sites that are relatively invariant across tissues, even if Meis affinity for them in individual cells is lower than for tissue-specific binding sites. 

Regarding motifs, the lack of \ac{OCTA} in eye peaks is to be expected, since there is no Hox expression in the eye, but for \ac{HSPC} peaks it is puzzling. According to the Gene Expression Commons (http://gexc.stanford.edu), a repository of hematopoietic gene expression patterns, several Hox genes are expressed in \acp{HSPC}: at least Hoxc6, Hoxa2, 3, 5 and 9 and Hoxb2 to 5 are expressed at this stage in hematopoietic development. We don't find any reason why Meis should not bind DNA in concert with them as we understand it does in the trunk. One possible explanation is the aforementioned bias inherent to our sample: if Meis-Hox binding has lower affinity than Meis monomeric binding but is more invariant across tissues, it could have stayed under the detection threshold in \textcite{Wilson2010}. However, there is no evidence supporting this explanation and in fact the DNA binding affinity of Hox-Meis heterodimers is higher than that of Meis monomers. One other difference is that in \acp{HSPC} Meis1 is the only expressed gene of the family, but again there is no reason to suspect that in our sample Meis2 is exclusively responsible for co-Hox binding.

%[Also expressed in HSCs: Meis1 but not 2, Pbx1 and 3, not Prep1 nor 2 (not really ubiquitous)]

Another aspect of the comparison that should be pointed out is that we do not find \ac{DECA} binding in the eye. Since Pbx is present there, there is no obvious reason why Meis should not bind in concert with it. Maybe heterodimeric Pbx-Meis binding is something that happens \textit{in vitro} but is just not common \textit{in vivo}, just as heterotrimeric Prep-Pbx-Hox seems to be.

The appearance in \ac{HSPC} peaks of the accessory motif  CWGSCWG supports the hypothesis that it is the recognition sequences for an hematopoietic transcription factor. The presence of CCCgCCC in a set that is not predominantly \ac{TSSA} argues against it appearing in Prep peaks purely because of their high association with promoters

The poly-purine motif present in all \ac{TALE} trunk peak subsets appears again in \acp{HSPC} (m2) and eye (m3) peaks, which suggests the possibility of it being a general requirement for Meis binding, since it seems to be the only constant pattern in Meis binding sites across tissues apart from the \ac{HEXA} motif. 

One conclusion to be drawn from the fact that Meis binding sites in different tissues are almost non-overlapping but share the many of the same sequence motifs is that presence of a direct binding target is insufficient to determine binding \textit{in vivo}. This is important because to date, we do not know what determines transcription factor binding: all known \acp{TF} have recognition sequences that appear many thousands of times in the genome but they only bind a subset of them.

The selection of actual targets from a pool of adequate candidates might be based on chromatin accessibility, interaction with additional cofactors, or sequence features that might affect fine double helix structural features, such as minor groove width. A recent study of the regulatory activity of short sequences from \textit{in vivo} binding sites has shown that, in the case of the homeodomain factor Crx, high GC content is as powerful a predictor of cis-regulatory potential as the presence of high quality recognition sites for Crx \parencite{White2013}. Since this study was carried out with reporter constructs, chromatin accessibility was not a factor and therefore can not be the only differentiator between bound and not bound potential binding sites. In the case of Meis, it is clear that the sequence patterns described in our work are not sufficient to explain \ac{GW} patterns of binding and their change across tissues. Maybe the key to predicting tissue-specific binding site occupancy lies in the large scale chromatin state. Alternatively, it could be dependent on subtler, more local sequence patterns that our present methods are unable to characterize. Time will tell.

\section{ejemplo(s?) de cherry picking}

%[lo que he pensado es meter aqui unos pocos casos concretos de patrones de picos interesantes para ilustrar el potencial del dataset]

Id2 (inhibitor of differentiation 2) has many \ac{TALE} binding sites in its vicinity, including a \ac{pbpr} peak in its promoter. Id2 represses myogenesis and it is in turn repressed by Rp58 (also known as Zfp238), a direct target of MyoD, to promote myogenesis (\cite{Yokoyama2009}). Rp58 has a \ac{pbpr} peak 2kb upstream of its TSS plus an hematopoietic Meis peak from \cite{Wilson2010}. It is suggestive that this system has already shown to be connected with the \ac{TALE} proteins (see introduction), which suggests a relatively widespread intertwining of \ac{TALE} function and MyoD. This example exemplifies the potential usefulness of the data we have generated to give insight into unexpected functions and relationships of the \ac{TALE} genes and to generate hypotheses. 

% *Id1 prep peak with HEXA and GC box motifs
