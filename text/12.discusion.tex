\chapter{Discussion}
\label{chp:discussion}

The results presented here represent the first \ac{GW} view of \ac{TALE} family proteins \acp{BS} in the vertebrate embryo. As such, they deepen our understanding of the target sequences and modes of action of these Hox cofactors, and raise interesting questions that will have to be addressed in further studies. 

\section{Binding Preferences}

Despite their very similar binding behaviour \textit{in vitro}, Prep and Meis have shown radically different binding sequence preferences \textit{in vivo} in our study. 

\subsection{Location}

This study is the first detailed view and comparison of \ac{TALE} proteins. Only a few \textit{in vivo} target sites were known previously. These have been invariably been regulatory sites located near genes. In contrast, our unbiased \ac{GW} search has revealed striking differences in \ac{TALE} protein promoter and enhancer occupancy. 

Meis binds thousands of extremely conserved gene-remote regions, a high proportion of which show enhancer chromatin marks in comparable cells. It seems reasonable to conclude that many of these sites are enhancers, at least some of which are inactive but 'poised'. Poised enhancers have been suggested to represent the range of responses available to a cell (\cite{Creyghton2010}) and could represent a developmental path restriction by the \ac{TALE} system which would be later narrowed down by the participation in \ac{TALE}-Hox complexes of subsets of Hox proteins available in different cells. 

Along this line, the Pbx-Meis heterodimer has been proposed to function as a 'pioneer' factor that can bind heterochromatin and recruit MyoD and the estrogen receptor to some of their targets, such as myogenin or M-cad (\cite{Berkes2004,Magnani2011}). It is notable though that our \ac{CHIP}-seq shows binding of Pbx and Prep at the M-cadherin promoter, but not binding of Meis. [discutir]

% M-cad expressed E11.5 in proximal limb

\subsection{Sequences}



\section{TALE Proteins as Hox cofactors}

The surprising finding that the most frequent sequence motif in Meis peaks is \ac{OCTA}, the Pbx-Hox target site, rather than \ac{HEXA}, the MEis monomer binding site, suggests that most Meis binding sites are trimeric Pbx-Hox-Meis complex binding sites in which Meis does not contact DNA directly. 

\subsection{OCTA Variant Preferences}



\subsubsection{New Hox Target Sequence}

\subsection{Restricted Binding in the Hox Clusters}

%\section{TALE Proteins as Transcription Factors}


\section{Additional Binding Partners}

We have discovered three new sequences not shown previously to be involved in \ac{TALE} protein binding. 

The sequence CWGSCWG is the only Pbx core motif we found. It appears in the \ac{meise} and \ac{pbxe} subsets and a similar one appears in \ac{mepb} peaks. It is likely to represent the binding sequence for an unknown Pbx or Pbx-Meis cofactor. It is interesting that highly similar sequences have been reported previously as unidentified motifs in several hematopoietic factor ChIP-seq experiments: An Scl/Tal1 ChIP-seq in immature erythocytes (\cite{Kassouf2010}), a 10-factor ChIP-seq in the hematopoietic progenitor cell
line HPC-7 (\cite{Wilson2010}), and a \ac{GW} analysis of Gata1, Gata2, Runx1, Fli1, and Scl/Tal1 binding in primary human megakaryocytes (\cite{Tijssen2011}). The only common factor to these three studies is Scl/Tal1. Scl/Tal1 is a \ac{BHLH} protein and thus binds an E-box. Although this novel motif resembles an E-box, the binding preferences of Scl/Tal1 are well characterized and do not include this variant, in which there is an additional nucleotide between the two halves of a canonical E-box. % TGIF? paper de binding alterno y efecto opuesto.

Scl/Tal1 can interact with Gata1 to specify erythroid cells (\cite{ref) and we have found an Scl/Tal1-related motif but not any motif resembling Gata1. Considering that Meis is necessary for megakaryocyte generation in the mouse embryo , it is tempting to speculate that Scl might be recruited by Gata1 in erythroid cells (\cite{Anguita2004}) and Meis1 in megakaryocytes. These two cell types share a precursor cell, the \ac{MEP} (\cite{ref}), so this could represent a protein recruitment-based developmental switch. Scl/Tal1 mutants die at 9.5 due to complete lack of blood (\cite{ref}), but deletion in adult mice shows it is necessary for maintenance only of the erythroid and megakaryocytic lineages (\cite{Mikkola2003}). It is remarkable that a DNA-binding deficient form of Scl rescues the early hematopoiesis-related lethality and allows the embryo to progress to E14.5, when it dies with incomplete maturation of erythrocytes (\cite{Porcher1999, Kassouf2008}). Scl is also expressed in the neural tube in a \ac{DV}-restricted manner {Muroyama2005}, as is Meis.


%Porcher 1999: scl can rescue most of the phenotype without its DNA binding domain

% Wilson2009: Scl/Tal1 binding shows Meis motifs in Hematopoietic precursor cell-7 (HPC-7) line
% Kassouf2010: immature, Ter119− erythroid cell populations derived from day E12.5 wild-type (Tal1WT/WT) fetal livers 

GGGGcGGGG: Klf?

Poly-purine tract - Pu.1? triplexes?
% FANCM, helicase, found in meis chip-prot, promotes dissociation of DNA triplexes


CCAAT