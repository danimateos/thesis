\chapter{Conclusions}
\label{chp:conclusions}

\begin{itemize}

  \item Meis and Prep have largely non-overlapping genomic binding sequences. Pbx1 overlaps much more with Prep than with Meis. 
  
  \item While Prep binds mainly at \acp{TSS}, specially in concert with Pbx, Meis \acp{BS} are extremely conserved and distributed evenly along the genome, with the caveat that exons are less likely to contain Meis \acp{BS}. A large proportion of Meis \acp{BS} are enhancers. 
  
  \item The sequences at sites bound by Meis, Prep and Pbx suggest Prep tends to bind as a heterodimer with Prep, without participation of Hox proteins. In contrast, Meis tends to bind DNA in concert with Hoxes, probably as a non-DNA-binding partner in trimeric Meis-Hox-Pbx complexes.
    
  \item Meis and Prep often regulate the same genes through different cis-regulatory sequences. In many cases, they regulate transcription of a given gene in opposite directions.  
  
  \item The Hox clusters are controlled colinearly by the \ac{TALE} cofactors. DNA binding in the clusters is dominated by Meis peaks and is restricted to the part of each cluster 3' to the 9\textsuperscript{th} paralog.
  
  \item The transcriptional effect on the Hox clusters is in opposite directions between Meis and Prep and between the 5' and 3' parts. Surprisingly, the largest effect is that seen in \ac{prepi} embryos on the 5' part of the clusters, in which neither Meis nor Prep bind.
   
  \item At least some Hox proteins are able to bind the novel sequence TGATGAAT.

\end{itemize}