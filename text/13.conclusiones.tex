\chapter{Conclusions}
\label{chp:conclusions}

\begin{itemize}

  \item Meis and Prep have largely non-overlapping genomic binding sequences. Pbx1 overlaps much more with Prep than with Meis. 
  
  \item While Prep binds mainly at \acp{TSS}, specially in concert with Pbx, Meis \acp{BS} are extremely conserved and distributed throughout the genome and do not show a particular relationship with transcriptional units, with the caveat that exons are less likely to contain Meis \acp{BS}. A large proportion of Meis \acp{BS} are enhancers. 
  
  \item The sequences at sites bound by Meis, Prep and Pbx1 suggest Prep tends to bind as a heterodimer with Pbx, without participation of Hox proteins. In contrast, Meis tends to bind DNA in concert with Hoxes, probably as a non-DNA-binding partner in trimeric Meis-Hox-Pbx complexes.
    
  \item Meis and Prep regulate some of the same genes through different cis-regulatory sequences. In many cases, they regulate transcription of a given gene in opposite directions.  
  
  \item TALE factor binding landscape subdivides Hox clusters in two domains. DNA binding in the clusters is dominated by Meis peaks and is restricted to the part of each cluster 3' to the 9\textsuperscript{th} paralog.
  
  \item The transcriptional effect on the Hox clusters is colinear with chromosomic position-- in opposite directions between Meis and Prep and between the 5' and 3' parts. Surprisingly, the largest effect is that seen in \ac{prepi} embryos on the 5' part of the clusters, in which we detect no binding of \ac{TALE} proteins.
   
  \item At least some Hox proteins are able to bind the novel sequence TGATGAAT.

\end{itemize}

\chapter{Conclusiones}
\selectlanguage{spanish}
\begin{itemize}

  \item Los sitios de unión al ADN de Meis y Prep apenas solapan. Los sitios de unión de Pbx1 solapan mucho más con los de Prep que con los de Meis.  %Meis and Prep have largely non-overlapping genomic binding sequences. Pbx1 overlaps much more with Prep than with Meis. 
  
  \item En contraste con Prep, que se une al ADN mayoritariamente en promotores especialmente en combinación con Pbx1, los sitios de unión de Meis están distribuidos por el genoma sin una relación general con unidades transcripcionales. Cabe el matiz de que los exones están deplecionados de picos Meis. Una proporción importante de picos Meis son enhancers. %While Prep binds mainly at \acp{TSS}, specially in concert with Pbx, Meis \acp{BS} are extremely conserved and distributed evenly along the genome, with the caveat that exons are less likely to contain Meis \acp{BS}. A large proportion of Meis \acp{BS} are enhancers. 
  
  \item Las secuencias de bases que encontramos en sitios de unión de Meis, Prep y Pbx1 sugieren que Prep une ADN casi siempre en complejo con Pbx, sin participación de proteínas Hox. En cambio, Meis tiende a unir ADN en complejo con Hoxes, probablemente como miembro de complejos Pbx-Meis-Hox pero sin unir ADN. %The sequences at sites bound by Meis, Prep and Pbx suggest Prep tends to bind as a heterodimer with Prep, without participation of Hox proteins. In contrast, Meis tends to bind DNA in concert with Hoxes, probably as a non-DNA-binding partner in trimeric Meis-Hox-Pbx complexes.
    
  \item Meis y Prep regulan algunos de los mismos genes a través de secuencias reguladoras en \textit{cis} distintas. En muchos casos, regulan la transcripción en sentidos opuestos. %Meis and Prep often regulate the same genes through different cis-regulatory sequences. In many cases, they regulate transcription of a given gene in opposite directions.  
  
  \item El paisaje de unión de proteínas \ac{TALE} divide los clusters Hox en dos dominios. La unión de proteínas \ac{TALE} a los clusters está dominada por Meis y está restringida a la parte de cada cluster situada 3' del parálogo 9.%The Hox clusters are controlled colinearly by the \ac{TALE} cofactors. DNA binding in the clusters is dominated by Meis peaks and is restricted to the part of each cluster 3' to the 9\textsuperscript{th} paralog.
  
  \item El efecto transcripcional de Meis y Prep en los genes Hox es colineal con la posición cromosómica de los genes, y en sentidos opuestos entre Meis y Prep. Sorprendentemente, el efecto más marcado lo observamos en la parte 5' de los clusters, en la que no detectamos unión de proteínas \ac{TALE}.%The transcriptional effect on the Hox clusters is in opposite directions between Meis and Prep and between the 5' and 3' parts. Surprisingly, the largest effect is that seen in \ac{prepi} embryos on the 5' part of the clusters, in which neither Meis nor Prep bind.
   
  \item Al menos algunas proteínas Hox son capaces de unir la secuancia de unión TGATGAAT, no descrita previamente. %At least some Hox proteins are able to bind the novel sequence TGATGAAT.

\end{itemize}
\selectlanguage{english}