\chapter{Results \& Discussion}
\section{Cataloging, describing Meis, Prep, Pbx1 ChIP-seq peaks}

\subsection{\# reads, sequencing, mapping info}
		Chromatin Immunoprecipitation was performed in the dissected trunks of embryonic stage E11.5 C57BL/6 mice.

\subsection{Peak calling}

\subsection{Meis, Prep, Pbx1 overlap}

\subsubsection{Triple peak Chip-reChip}

\subsection{Distance to TSSs}

\subsection{TSSA, IG, CI}

\subsubsection{exon/intron Meis ratio is skewed, others not.}

\subsection{Conservation}

\subsection{histone marks}


\section{Sequence analysis}

\subsection{Motif discovery stats, core motifs}

\subsection{distribution in different subsets}

\subsection{Comparison with expected distribution of motifs}

\subsection{Hox motif variants, distribution}

\subsection{uncore motifs}

\section{Transcriptional targets}

\subsection{\# reads, sequencing, mapping info}

\subsection{attempting to correlate}

\subsection{logistical model (?)}

\section{Hox Clusters}

\subsection{Distribution in the Hox clusters}

\subsection{AP ChIP}

\subsection{previously described reg sites}

\subsection{Hox RNAseq}


[por colocar:

- comparacion detallada con Meis ChIP-seq hematop {Wilson et al. 20101, Huang et al. 2012}
- Chip-proteomics
]
