\chapter{Matherials \& Methods}
\label{chp:methods}

\section{ChIP-seq}
\subsection{Chromatin Immunoprecipitation}

Chromatin immunoprecipitations were performed using standard methods with anti-Prep1/2 antibody (N15, Santa Cruz Biotechnology, Santa Cruz, USA), anti-Pbx1 antibody (4342) (Cell signaling technology, Beverly, USA) and a mix of anti-Meis antibodies (K830, recognizing Meis1a and Meis2a isoforms, and K844, recognizing Meis1a and Meis1b isoforms; both produced at the CNIC, Madrid, Spain).

A single-cell suspension was prepared from E11.5 mouse embryonic body trunks (total embryo without head, tail and legs) by crushing them against a cell strainer. Approximately 5x10\textsuperscript{7} cells were used for each immunoprecipitation. Cells were cross-linked in complete medium (10\% FBS) containing 1\% formaldehyde for 10 min, and the reaction was terminated by addition of 125 mM glycine. Fixed cells were washed three times (5 min each) in cold PBS and lysed in LB1 buffer containing 0.5\% NP-40 and 0.25\% triton X-100. Nuclei were then washed in LB2 buffer (containing 10mM Tris-HCl pH=8 and 200mM NaCl) to remove detergents and resuspended in LB3 buffer, containing 0.1\% Na-deoxycholate and 0.5\% N-lauroylsarcosine. Chromatin was sonicated by 5 x 30 sec cycles at 30\% of the maximum power of a Branson 450 sonicator to generate 100-400 bp chromatin fragments. After clearing by centrifugation, sonicated chromatin was incubated with antibody-bound protein A-conjugated magnetic beads (Invitrogen, Carlsbad, USA). For each IP we used 10 ug antibody. Rabbit IgG IP was performed as negative control. After overnight immunoprecipitation at 4$^\circ$C the bound complexes were washed twice in WB1 (50 mM Hepes-KOH pH 7.5, 140 mM NaCl, 1 mM EDTA, 1\% Triton-X100, 0.1\% Na-doexycholate), twice in WB2 (50 mM Hepes-KOH pH 7.5, 500 mM NaCl, 1 mM EDTA, 1\% Triton-X100, 0.1\% Na-doexycholate) and twice in LiCl WB (10 mM Tris-Cl pH 8.0, 250 mM LiCl, 0.5\% NP-40, 0.5\% Na-deoxycholate, 1 mM EDTA). Immunoprecipitated complexes were eluted from the beads by incubating for 30 min in EB (2\% SDS in TE) at 37$^\circ$C. The eluted material was reverse cross-linked at 65$^\circ$C overnight and incubated for 1 h at 55$^\circ$C with proteinase K. The obtained material was extracted with phenol-chloroform and ethanol-precipitated. After RNAse treatment, the DNA was purified with a \ac{PCR} purification kit (Qiagen, Netherlands). About 10 ng of immunoprecipitated DNA were processed for sequencing.

\subsection{Sequencing and Alignment}

Chromatin-immunoprecipitated DNA was sequenced using an Illumina GAII analyzer. Single-end 36bp reads (~120M for Meis and ~40M for Prep) were first mapped with BWA software (\cite{Li2009}) against mm9 version of the mouse genome.

\subsection{Peak Calling}

Peak-calling was performed with the \ac{PICS} algorithm (\cite{Zhang2011}). The algorithm was used to identify genomic regions with a high density of reads (peaks). These regions are an indicator of the enrichment of immunoprecipitated DNA fragments for the \ac{TF} of interest. \ac{PICS} has been demonstrated to perform well against other peak-calling algorithms such as \ac{MACS} (\cite{Zhang2008}) and CisGenome (\cite{Ji2011}).

\section{ChIP-re-ChIP}

For ChIP-reChIP the complexes were immunoprecipitated with anti-Pbx1 Ab, anti-Meis (a 1:1 mix of anti-Meis1 and anti-Meisa) Ab or rabbit IgG (negative control) and washed twice in WB1 and twice in WB2. They were then eluted from the beads by incubating for 30 min at 37 $^\circ$C in reChiP elution buffer (2\% SDS in TE supplemented with 15 mM DTT). The obtained material was diluted 20-fold in ChIP dilution buffer (0.01\% SDS, 1.1\% Triton X-100, 1.2 mM EDTA, 16.7 mM Tris–HCl, pH 8.0, 167 mM NaCl) and used for the second IP with anti-Pbx1, anti-Prep1/2, anti-Meis antibodies or rabbit IgG. The second immunoprecipitation was performed as described above. \ac{PCR} amplification of target DNA peaks was conducted with the following oligonucleotide primers: [rellenar]

\section{AP-ChIP}

For comparison of Meis site occupancy between positions along the anterior to posterior axis of the embryo, E11.5 embryos were dissected according to the diagram in  Figure [referenciar figura Hoxes aqui]. ChIP was performed as described on nuclear extracts from each of the portion and the resulting chromatin uncrosslinked for \ac{PCR} testing.

\subsection{Primers}

Proportionality of product amount to input concentration was tested along a range of input DNA amount and number of cycles. 30 to 35 \ac{PCR} cycles were chosen as the optimal proportionality point and used for occupancy testing. Some primers were found not to produce a satisfactory proportionality between input and output and results discarded. [seqs de los primers]

\subsection{PCR Conditions}

Primers were designed for all Meis peaks in the Hox clusters using primer-BLAST and targeting amplification conditions of 60$^\circ$C and product length around 100pb. 

\section{ChIP-chip [preguntar a Dima si lo puedo usar]}

Chromatin immunoprecipitations of mouse thymocyte lysates were performed as described above with anti-Prep1/2 antibody (N15, Santa Cruz Biotechnology, Santa Cruz, USA) and anti-Pbx2 antibody (G20, Santa Cruz Biotechnology, Santa Cruz, USA). A single-cell suspension was prepared from thymuses from 6-8-wk-old mice (wt C57B6 strain) by crushing them against a cell strainer. Approximately $5x10^7$ cells were used for each immunoprecipitation. After immunoprecipitation and purification, DNA was amplified using a Whole Genome Amplification (WGA1) kit (Sigma-Aldrich, St-Louis, USA). About 3-5 ug of the amplified material was used to prepare the probe for hybridization to the Nimblegen mouse RefSeq promoter array (C4222-00-01)(Nimblegen-Roche, Germany). Probe preparation, hybridization reactions and data analysis were performed according to the manufacturer’s recommendations. Peak enrichment was analyzed by Model-based analysis of two-color arrays (MA2C) (cistrome.org) using the following parameters: bandwidth 300, max gap 250, min probes 5, threshold method P value, value, $10^{-6}$, normalization method robust, C value 2, mm9 assembly.

\section{RNA extraction}

For RNA-seq, total RNA was purified from whole E11.5 mouse embryos (\ac{prepi}, Meis1 \ac{KO} and wt control littermates of each). mRNA was purified and the library for Illumina chromatin sequencing prepared according to the Illumina recommendations.

\subsection{Sequencing, Alignment and Fold Change estimation}

mRNA samples were sequenced using the paired-end 50bp protocol. Reads (~7M per sample) were mapped and transcript expression estimated using RSEM (\cite{Li2011}). This program aligns the reads against a set of predefined transcripts (in our case mouse ensemble 63 genebuild) and uses an expectation maximization algorithm to assign reads probabilistically to one of the isoforms of a given gene. The quantification results from RSEM were then analyzed with the Bioconductor package DESeq (\cite{Anders2010}), which fits a negative binomial distribution to estimate technical and biological variability.


\section{EMSA}

Nuclear extracts were isolated from cells prepared from E11.5 mouse embryonic body trunks (wt CD1 strain) as described (Longobardi and Blasi, 2003). Briefly, cells were washed twice with cold PBS, collected in 500 \textmu l cold buffer A (10 mm HEPES, pH 7.9, 10 mm KCl, 1.5 mm MgCl2, 0.5 mM DTT, proteinase inhibitor cocktail), left for 10 min on ice, and lysed by adding Triton X-100 to a final concentration of 0.3\%. Nuclear extracts were prepared by resuspending pelleted nuclei in 100 \textmu l buffer C (20 mm HEPES, pH 7.9, 25\% glycerol (v/v), 0.42 m NaCl, 1.5 mm MgCl2, 0.5 mM DTT, 0.5 mm EDTA, proteinase inhibitor cocktail) for 30 min on ice. The extract was cleared by centrifugation.
EMSA reactions were performed in 20 \textmu l reaction mix containing 10 mM Tris-HCl (pH 7.6), 0.5 mM EDTA 0.5, 0.5 mM EGTA, 5\% glycerol buffer, 80 mM NaCl, 1\textmu g poly-dIdC, 1 mM DTT, 5 \textmu g nuclear extract and 30,000 cpm of the 32P-labeled probe. When indicated, we added unlabeled probe in 50X excess or specific antibodies (4 \textmu g) against Prep1/2 (N15, Santa Cruz Biotechnology, Santa Cruz, USA), Meis (1:1 mix of K830 and K844) or Pbx1 (P20, Santa Cruz Biotechnology, Santa Cruz, USA). Reactions were carried out for 30 min at room temperature, and the complexes resolved by 5\% non-denaturing PAGE. The gel was dried and scanned in a phosphorimager. 
The following double-stranded oligonucleotides were used, in addition to mutant versions as indicated in figures: and text:
5’-AGGGAAGAGCCTGACAGATGACAGTTTCGAAAAA-3’ (Meis Peak 586) -- HEXA
5’-CAAATAACTGATTGATTGCGGTCGAGGCACATTG-3’ (Meis Peak 313) – OCTA
5’-GGCCTCGTGATTGACAGGCTCGCCG-3’- DECA
5’-ATGCTGTGACAGTGATAAATGACGGTGCAGAA-3’ (Meis  Peak 1) -- OCTA + HEXA
5’-AATTTAAGCAGTGATGAATGAGCTCGGCTG-3 (Meis Peak 595)
5’-TTTGGGTGACAAAGATGAATGGTCTATTGT-3’ (Meis Peak 9)
5’-GACAACCTCGCCTGTGATTGACCCCTGGAGTGG-3’ (Meis Peak 54)
5’-AGGGGCCCGTGATTGACAGGCTGAACTACAGACT-3’ (Prep Peak 1440)
5’-TTGCTGACAACTGACTGATAAATTATTTTCTGCT-3’ (Meis Peak 857)

\section{Alleles used for LOF}

\section{ChIP proteomics}

\section{Bioinformatics}

\subsection{Motif Discovery, FIMO}

For the identified peaks, \textit{de novo} motif discovery was run to identify consensus sequences enriched in the selected regions versus the whole genome using rGADEM (\cite{Li2009a}). While similar in its model to previous de novo motif finder algorithms such as MEME (\cite{Bailey2006}), GADEM performs better in large ChIP-seq experiments. Interestingly, rGADEM can identify dimer motifs located close to each other. The algorithm reports probability weight matrices (PWM) summarizing the consensus sequences identified in the peaks. (iii) We identified, interpreted and annotated de novo motifs using MotIV (motif identification and validation). MotIV compares the PWM of the identified de novo motifs with those in the JASPAR TF database (http://jaspar.genereg.net) and provides a significance level for the similarity. MotIV can furthermore report the distribution of distances between pairs of motifs and the percentage of co-localized motifs within the same peak, which might suggest cooperation between TFs. Steps (ii) and (iii) of the pipeline were run for the peaks identified for each TF separately and also for peaks found for two or more TFs. 

Individual instances of the core motifs within all peaks were searched with a local install of the FIMO (Find Individual Motif Ocurrences) program from the MEME suite (\cite{Bailey2009}) with default parameters (individual p-value cutoff $10^{-4}$). The resulting files were parsed and the data added to the peak tables using a custom Python script. 



\subsection{Exploratory Data Analysis}


Peak overlapping, correlation with RNA-seq data and conservation data aggregation were performed on the Galaxy bioinformatics platform (\cite{Blankenberg2010a, Goecks2010}). The set of peak coordinates for each factor was intersected with the sets for the other factors, the nearest transcript expression data were added to the tables, and the average conservation value (measured as the PhastCons 30-way vertebrate score (\cite{Siepel2005}) was aggregated over the length of each peak with the aggregate genomic scores tool.

GC percentage count and literal motif match finding were performed using custom Python scripts (http://www.python.org/) run over the fasta sequence of the peaks.

Genomic profiling of peaks was performed using custom Python scripts and the CEAS (Cis-regulatory Element Annotation System) tool (\cite{Shin2009}) within the Cistrome package (\cite{Liu2011}). Peak association with regulated genes was estimated by calculating peak density per megabase in all Ensembl v63 nuclear genes, their promoters and close intergenic regions and comparing with the corresponding values for genes up- or down-regulated in Meis1 \ac{KO} and \ac{prepi} embryos. Peak overlap with histone modification data was established by considering 3000pb around the summit of the histone mark peaks and intersecting with transcription factor peaks.

\section{Statistical Testing}

\subsection{Random Expectations}

For \ac{OCTA} motif variants, we considered the expected number of instances of a motif containing i G/Cs and j A/Ts to be $I_{variant} = N x {\left(\frac{P(GC)}{2}\right)}^i x {\left(\frac{P(AT)}{2}\right)}^j$, where I is the number of instances expected, N is the number of bases in the collection and GC and AT are the GC and AT fractions in the collection.

\subsection{Tests Used}

We estimated the significance of OCTA motif variants with a Poisson test in which $\lambda$ equals the random exxpected number of instances.

Overlap between \ac{TALE} factors and Hoxes was tested with $\chi^2$ tests of independence applied to the relevant contingency tables.

